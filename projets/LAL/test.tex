\setlength{\LTpost}{0mm}
\begin{longtable}{|p{8cm}|p{0.5cm}|p{2.5cm}|p{2cm}|p{2cm}|} %Clé de l'ajustement
\toprule
 &  &  & \multicolumn{2}{c}{\textbf{Sexe des individus}} \\ 
\cmidrule(lr){4-5}
\textbf{Variables} & \textbf{N} & \textbf{Overall}, N = 155\textsuperscript{\textit{1}} & \textbf{F}, N = 75\textsuperscript{\textit{1}} & \textbf{M}, N = 80\textsuperscript{\textit{1}} \\ 
\midrule\addlinespace[2.5pt]
\textbf{Nb blastes circulants au diagnostic, Giga/L} & 155 & 8 (1, 41) & 5 (0, 25) & 14 (2, 55) \\ 
\textbf{Nb globules blancs au diagnostic, Giga/L} & 155 & 19 (6, 65) & 14 (5, 36) & 26 (9, 74) \\ 
\textbf{PS\textsuperscript{\textit{2}}} & 154 &  &  &  \\ 
    0 &  & 53 (34\%) & 19 (25\%) & 34 (43\%) \\ 
    1 &  & 75 (49\%) & 43 (57\%) & 32 (41\%) \\ 
    2 &  & 24 (16\%) & 12 (16\%) & 12 (15\%) \\ 
    3 &  & 2 (1.3\%) & 1 (1.3\%) & 1 (1.3\%) \\ 
\textbf{Atteinte de la rate} & 155 & 22 (14\%) & 6 (8.0\%) & 16 (20\%) \\ 
\textbf{Atteinte du foie} & 155 & 12 (7.7\%) & 6 (8.0\%) & 6 (7.5\%) \\ 
\textbf{Atteinte du médiastin} & 147 & 1 (0.7\%) & 0 (0\%) & 1 (1.4\%) \\ 
\textbf{Nb blastes moelle, Giga/L} & 155 &  &  &  \\ 
    1 &  & 151 (97\%) & 73 (97\%) & 78 (98\%) \\ 
    2 &  & 4 (2.6\%) & 2 (2.7\%) & 2 (2.5\%) \\ 
\textbf{\% blastes moelle} & 151 & 88 (80, 94) & 88 (80, 93) & 88 (79, 94) \\ 
\textbf{Taux d'hémoglobine, g/dL} & 155 & 10.50 (8.65, 12.80) & 10.30 (7.75, 11.85) & 10.80 (9.18, 13.23) \\ 
\textbf{Nb de neutrophiles, Giga/L} & 154 & 2.5 (1.1, 5.8) & 2.3 (1.3, 5.3) & 3.3 (1.0, 6.8) \\ 
\textbf{Nb de lymphocytes, Giga/L} & 152 & 3.1 (1.8, 5.7) & 2.9 (1.6, 5.7) & 3.3 (2.2, 5.6) \\ 
\textbf{Nb de monocytes, Giga/L} & 152 & 0.15 (0.00, 0.64) & 0.14 (0.02, 0.40) & 0.16 (0.00, 0.89) \\ 
\textbf{Nb de plaquettes, Giga/L} & 155 & 68 (29, 147) & 52 (27, 135) & 78 (31, 155) \\ 
\bottomrule
\end{longtable}
\begin{minipage}{\linewidth}
\textsuperscript{\textit{1}}Médiane (Q1, Q3) ou Fréquence (\%)\\
\textsuperscript{\textit{2}}Échelle de statut de performance-ECOG ; \textbf{0}: Patient entièrement actif, capable d'effectuer les mêmes activités pré-morbides sans restriction; \textbf{1}: Patient restreint dans ses activités physiques, mais ambulatoires et capables d'effectuer des activités légères ou sédentaires, par ex. : travaux ménagers légers ou tâches administratives; \textbf{2}: Patient ambulatoire et capable de s'occuper de lui, mais incapable d'effectuer des activités. Debout \textgreater{} 50\% de la journée; \textbf{3}: Patient capable de soins limités, alité ou au fauteuil \textgreater{} 50\% de la journée.\\
\end{minipage}

